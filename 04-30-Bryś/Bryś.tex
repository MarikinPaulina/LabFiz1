\documentclass[a4paper,10pt]{article}
\usepackage[margin=1in]{geometry}
\usepackage{polski}
\usepackage[utf8x]{inputenc}
\usepackage[unicode]{hyperref}
\usepackage{amssymb}
\usepackage{xifthen}
\usepackage[fleqn]{amsmath}
\usepackage{todonotes}
\usepackage{graphicx}
\usepackage{float}
\usepackage{fullpage}
\usepackage{epstopdf}
\usepackage{multirow}
\usepackage{subfig}
\usepackage[europeanresistors,americaninductors]{circuitikz}
\usetikzlibrary{patterns}
\newcommand{\withtodo}{0}


\def\arraystretch{1.2}


\begin{document}

\begin{table}
  \centering
  \def\arraystretch{1.5}
    \begin{tabular}{|l|l|l|l|} \hline
    Wydział:           & \multicolumn{2}{l|}{Dzień:Poniedziałek 14-17}    &Zespół:  \\
    Fizyki             &    \multicolumn{2}{l|}{Data: 20.03.2017}         &8             \\\hline
    Imiona i nazwiska: &Ocena z przygotowania:  &Ocena ze sprawozdania:   &Ocena końcowa: \\
    Marta Pogorzelska  &                        &                         &                \\
    Paulina Marikin    &                        &                         &\\\hline
    \multicolumn{2}{|l|}{Prowadzący:                 } &\multicolumn{2}{l|}{Podpis:             }  \\\hline
  \end{tabular}
\end{table}

\title{Ćwiczenie 30:\\Odbicie światła od powierzchni dielektryka}
\date{}
\maketitle

\section{Cel badań}
Celem doświadczenia było zweryfikowanie poprawności prawa Snella i prawa Malusa oraz wyznaczenie
kąta granicznego, kąta Brusnela i wspówłczynnika załamania badanego dielektryka.

\section{Wstęp teoretyczny}
\subsection{Prawo Snella}
Fala elektromagnetyczna na granicy ośrodków ulega dwóm zjawiską: załamaniu i odbiciu, gdzie fala załamana jest częścią fali, która zmieniła
ośrodek, zaś fala odbita częścią pozostałą w pierwotnym ośrodku. Kąty pod jakimi rozchodza się te fale (mierzone do normalnej - osi prostopadłej do płaszczyzny odbicia)
 są ze sobą powiązane przez prawo Snella:
\begin{equation}
    n_1\sin{\alpha} = n_2\sin{\beta}
\end{equation}
Kąt $\aplha$ jest kątem odbicia równym kątowi padania, $\beta$ to kąt załamania, zaś $n_1$ i $n_2$ to współczynniki załamania definiowane $n = \frac{c}{v}$
,gdzie v - prędkość fali elektromagnetycznej w danym ośrodku. Po przekształceniu
\begin{equation}
n_2 = n_1 \frac{\sin{\alpha}}{\sin{\beta}}
\end{equation}
można na podstawie prawa Snella wyznaczyć eksperymentalnie współczynnik załamania danego ośrodka.
\subsection{Kąt Brewstera}

\subsection{Kąt graniczny}

\subsection{Prawo Malusa}






\section{Opis układu i metody pomiarowej}

\paragraph{}...

\section{Wyniki pomiarów}...

\section{Analiza niepewności}

\section{Wnioski}
\paragraph{}...


\end{document}
