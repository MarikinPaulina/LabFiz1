\documentclass[a4paper,10pt]{article}
\usepackage[margin=1in]{geometry}
\usepackage{polski}
\usepackage[utf8x]{inputenc}
\usepackage[unicode]{hyperref}
\usepackage{amssymb}
\usepackage{xifthen}
\usepackage[fleqn]{amsmath}
\usepackage{todonotes}
\usepackage{graphicx}
\usepackage{float}
\usepackage{fullpage}
\usepackage{epstopdf}
\usepackage{multirow}
\usepackage{subfig}
\usepackage[europeanresistors,americaninductors]{circuitikz}
\usetikzlibrary{patterns}
\newcommand{\withtodo}{0}

\def\arraystretch{1.2}

\begin{document}

\begin{table}
  \centering
  \def\arraystretch{1.5}
    \begin{tabular}{|l|l|l|l|} \hline
    Wydział:           & \multicolumn{2}{l|}{Dzień:Poniedziałek 14-17}    &Zespół:  \\
    Fizyki             &    \multicolumn{2}{l|}{Data: 20.03.2017}         &8             \\\hline
    Imiona i nazwiska: &Ocena z przygotowania:  &Ocena ze sprawozdania:   &Ocena końcowa: \\
    Marta Pogorzelska  &                        &                         &                \\
    Paulina Marikin    &                        &                         &\\\hline
    \multicolumn{2}{|l|}{Prowadzący:                 } &\multicolumn{2}{l|}{Podpis:             }  \\\hline
  \end{tabular}
\end{table}

\section{Cel badań}
\paragraph{Badanie wpływu rodzaju i grubości materiału na osłabienie
promieniowania gamma przy przechodzeniu przez materię oraz wyznaczenie współczynnika osłabienia przykładowych materiałw.}

\section{Wstęp teoretyczny}
\paragraph{Promieniowanie gamma, zwane inaczej promieniowaniem elektromagnetycznym,
jest jednym z trzech rozpadów promieniotwórczych występującym w przyrodzie.
Zwykle jest ono obecne podczas pozostałych dwóch przemian: alfa i beta, i polega
na wyzbyciu się przez jądro atomu nadmiaru energii, tzw. energii wzbudzenia.
W przeciwieństwie do rozpadów alfa i beta podczas promieniowania gamma nie następuje zmiana liczby neutronów ani protonów  w jądrze.}
\paragraph{Promieniowanie gamma przechodząc przez materię ulega osłabieniu.
Wpływ na to mają trzy zjawiska: rozpraszanie komptonowskie, zjawisko fotoelektryczne
oraz zjawisko tworzenia się par elektron-pozyton. Powodują one ubytek kwantów ɣ z
wiązki promieniowania, której natężenie po przejściu przez materię wyraża się wzorem:}

\begin{equation}
  I = I_0 * \exp(-\mu x)
\end{equation}
\paragraph{gdzie 	I0 – początkowe natężenie wiązki, miu – współczynnik osłabienia, x – grubość absorbenta.}
\paragraph{W celu wyznaczenia współczynnika miu potrzebna by była początkowa wartość natężenia wiązki.
 Bezpośredni pomiar rzeczywistej wielkości I0 prepararu jest niemożliwy, ponieważ w początkowym etapie badania
 (bez absorbenta) w domku osłonowym występuje nie tylko promieniowanie gamma, ale również niewielka ilość promieniowania beta.
 Ulega ono całkowitemu pochłonięciu przez materiał nawet przy zastosowaniu najcieńszego absorbenta.
 Z tego powodu w doświadczeniu będzie użyta metoda najmniejszych kwadratów, aby pominąć konieczność zmierzenia wartości
 I0.. Na podstawie serii pomiarów można wykreślić zależność liniową między grubości absorbenta a natężeniem wiązki
 promieniowania po przejściu przez niego. W tym celu należy jedynie zlogarytmować obie strony równania wzoru: }

\begin{equation}
  \ln(I) = -\mu x + \ln(I_0)
\end{equation}
\section{Metoda przeprowadzenia badań i pomiarów, materiały, aparatura}
\paragraph{Na początku należało włączyć aparaturę oraz odczekać 20 minut.
Przed włożeniem preparatu oraz absorbenta do domku osłonowego zmierzono pięciokrotnie
 tło układu. Następnie wykonano serię pomiarów natężenia wiązki promieniowania
 dla różnych grubości płytek, której umieszczano kolejno pomiędzy kolimatorami.
 Każdą serię wykonano dla trzech różnych rodzajów materiału: ołowiu, miedzi i aluminium.\\}

{Użyte przyrządy i materiały:}
\begin{itemize}
  \item domek osłonowy
  \item komputer mierzący pomiary
  \item kolimatory
  \item preparat $_{137}Cs$
  \item ołowiane i miedziane płytki grubości(około): 2, 5, 7, 10, 12, 15, 17 i 20 mm
  \item aluminiowe płytki grubości(około): 5, 10, 15, 20 mm
\end{itemize}

\section{Opracowanie pomiarów}

\begin{tabular}{|l|r|r|r|r|r|r|}
\hline
\multicolumn{7}{|c|}{Glin} \\\hline
{} &$\bar{x}$&S$\bar{x}$ &$\mu\bar{x}$&$\bar{x}(\mu\bar{x})$ &N($\mu_N$) &$\lnN(\mu_{lnN})$ \\\hline
20 &  20.041 &  0.0022   &  0.0062    &  20.041(6)           &  595(24)&  6.388(41) \\\hline
15 &  14.809 &  0.005    &  0.0078    &  14.809(8)           &  715(27)&  6.572(37) \\\hline
10 &  10.008 &  0.0024   &  0.0062    &  10.008(6)           &  731(27)&  6.594(37) \\\hline
5  &   5.020 &  0.003    &  0.0067    &   5.020(7)           &  754(27)&  6.625(36) \\\hline
\multicolumn{7}{|c|}{Miedź}                                         \\\hline
{} &$\bar{x}$&S$\bar{x}$ &$\mu\bar{x}$&$\bar{x}(\mu\bar{x})$ &N($\mu_N$) &$\lnN(\mu_{lnN})$ \\\hline
20 &  20.112 &  0.0024  &  0.0069 &  20.112(7) &  281(17) &  5.638(60) \\\hline
17 &  16.966 &  0.0025  &  0.0069 &  16.966(7) &  329(18) &  5.796(55) \\\hline
15 &  15.080 &  0.0020  &  0.0068 &  15.080(7) &  393(20) &  5.973(50) \\\hline
12 &  12.103 &  0.0025  &  0.0069 &  12.103(7) &  465(22) &  6.142(46) \\\hline
10 &  10.972 &  0.004   &  0.0074 &  10.972(7) &  484(22) &  6.182(45) \\\hline
7 &   7.045  &  0.0029  &  0.0071 &   7.045(7) &  555(24) &  6.318(42) \\\hline
5 &   4.849  &  0.004   &  0.0074 &   4.849(7) &  626(25) &  6.439(40) \\\hline
2 &   1.948  &  0.0028  &  0.0070 &   1.948(7) &  754(27) &  6.625(36) \\\hline
\multicolumn{7}{|c|}{Ołów}                                          \\\hline
{} &$\bar{x}$&S$\bar{x}$ &$\mu\bar{x}$&$\bar{x}(\mu\bar{x})$ &N($\mu_N$) &$\lnN(\mu_{lnN})$ \\\hline
20 &  19.953 &  0.004  &  0.0076 &  19.953(8) &   89(9)  &  4.48(11) \\\hline
17 &  16.910 &  0.0028 &  0.0070 &  16.910(7) &  122(11) &  4.80(9) \\\hline
15 &  14.840 &  0.004  &  0.0076 &  14.840(8) &  157(13) &  5.05(8) \\\hline
12 &  11.895 &  0.0029 &  0.0071 &  11.895(7) &  267(16) &  5.58(6) \\\hline
10 &   9.976 &  0.0025 &  0.0069 &   9.976(7) &  315(18) &  5.75(6) \\\hline
7 &   6.951  &  0.0022 &  0.0068 &   6.951(7) &  363(19) &  5.89(5) \\\hline
5 &   5.037  &  0.003  &  0.0073 &   5.037(7) &  504(22) &  6.22(4) \\\hline
2 &   1.845  &  0.0025 &  0.0069 &   1.845(7) &  711(27) &  6.56(4) \\\hline
\end{tabular}


\section{Analiza niepewności}

 \section{Wnioski}

\end{document}
