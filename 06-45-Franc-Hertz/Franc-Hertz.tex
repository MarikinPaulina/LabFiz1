\documentclass[a4paper,10pt]{article}
\usepackage[margin=1in]{geometry}
\usepackage{polski}
\usepackage[utf8x]{inputenc}
\usepackage[unicode]{hyperref}
\usepackage{amssymb}
\usepackage{xifthen}
\usepackage[fleqn]{amsmath}
\usepackage{todonotes}
\usepackage{graphicx}
\usepackage{float}
\usepackage{fullpage}
\usepackage{epstopdf}
\usepackage{multirow}
\usepackage{subfig}
\usepackage[europeanresistors,americaninductors]{circuitikz}
\usetikzlibrary{patterns}
\newcommand{\withtodo}{0}


\def\arraystretch{1.2}


\begin{document}

\begin{table}
  \centering
  \def\arraystretch{1.5}
    \begin{tabular}{|l|l|l|l|} \hline
    Wydział:           & \multicolumn{2}{l|}{Dzień:Poniedziałek 14-17}    &Zespół:  \\
    Fizyki             &    \multicolumn{2}{l|}{Data: 20.03.2017}         &8             \\\hline
    Imiona i nazwiska: &Ocena z przygotowania:  &Ocena ze sprawozdania:   &Ocena końcowa: \\
    Marta Pogorzelska  &                        &                         &                \\
    Paulina Marikin    &                        &                         &\\\hline
    \multicolumn{2}{|l|}{Prowadzący:                 } &\multicolumn{2}{l|}{Podpis:             }  \\\hline
  \end{tabular}
\end{table}


\section{Wstęp teoretyczny}
Poziomy energetyczne elektronów w atomie są skwantowane, czyli mogą przyjmować tylko określone, dyskretne wartości. Zmiana poziomu energetycznego z niższego na wyższy(wzbudzony)
może zajść tylko gdy elektron otrzyma porcję energii równą różnicy między tymi poziomami. James Franc i Gustaw Hertz w swoim doświadczeniu z 1913 roku potwierdzili ten fakt, czym
pomogli ugruntować kwantową teorię atomu. W swoim eksperymencie badali przewodzenie prądu przez elektrony w lampach wypełnionych gazowym neonem albo oparami rtęci. Zmiana prądu
związana ze zwiększaniem energii dostarczanej do elektronów nie zachodzi w takim przypadku monotonicznie, ale rośnie i maleje w równych przedziałach czasu. Dzieje się tak gdyż atomy
mogą pochłaniać energie rozpędzonych elektronów dopiero gdy osiągnie ona konkretną wartość odpowiadającą różnicy między dwoma poziomami energetycznymi.

\section{Opis układu i metody pomiarowej}
W skład układu pomiarowego dla lampy rtęciowej wchodzą:
\begin{itemize}
  \item lampa rtęciowa
  \item piec do ogrzania rtęci
  \item termopara z woltomierzem mierząca temperaturę rtęci
  \item wentylator
  \item zasilacz z możliwocią regulacji napięcia żarzenia, napięcia hamowania i napięcia przyspieszającego
  \item cztery woltomierze mierzące powyższe napięcia i napięcie anodowe
\end{itemize}
Układ pomiarowy dla noeonu jest podobny jednak nie zawiera pieca, termopary ani wentylatora, gdyż neon w temperaturze pokojowej jest w stanie gazowym. Zawiera zaś niewystępującą w
zestawie rtęci siatkę pozwalającą na ukierunkowanie strumienia elektronów.\\
W dowiadczeniu najpierw podgrzano rtęć do postaci gazowej. Następnie ustalono, stałe przez całe doświadczenie napięcie żarzenia i napięcie hamowania. Mierzone było napięcie
anodowe (będące wprost proporcjonalne do prądu anodowego) w zależności o zmienianego przez eksperymentatora napięcia przyspieszającego w zakresie od 0 do 30 voltów. Doświadczenie
dla neonu przebiegało analogicznie. Jedynymi różnicami był brak początkowego podgrzewania i zakres napięcia przyspieszającego od 0 do 70 voltów.

\section{Wyniki pomiarów}
Rtęć
\begin{tabular}{lrr}
\toprule
{} &  U[V] &  Ua[V] \\
\midrule
0  &   0.2 &   3.12 \\
1  &   0.5 &   3.18 \\
2  &   2.6 &   3.82 \\
3  &   3.6 &   3.01 \\
4  &   4.5 &   2.85 \\
5  &   5.5 &   3.75 \\
6  &   6.6 &   4.26 \\
7  &   7.6 &   4.11 \\
8  &   8.2 &   3.69 \\
9  &   9.0 &   4.15 \\
10 &   9.4 &   4.85 \\
11 &  10.6 &  10.60 \\
12 &  11.0 &  12.10 \\
13 &  11.3 &  13.77 \\
14 &  11.6 &  13.18 \\
15 &  12.5 &   6.42 \\
16 &  13.1 &   4.59 \\
17 &  13.3 &   4.31 \\
18 &  13.5 &   4.65 \\
19 &  14.2 &   5.69 \\
20 &  15.2 &  11.26 \\
21 &  15.7 &  16.60 \\
22 &  16.2 &  19.70 \\
23 &  16.4 &  19.89 \\
24 &  16.6 &  18.22 \\
25 &  17.6 &   7.65 \\
26 &  18.0 &   5.91 \\
27 &  18.3 &   5.17 \\
28 &  18.9 &   5.87 \\
29 &  20.0 &  12.68 \\
30 &  20.5 &  18.78 \\
31 &  20.8 &  23.37 \\
32 &  21.1 &  26.53 \\
33 &  21.4 &  27.97 \\
34 &  21.7 &  26.70 \\
35 &  22.7 &  12.85 \\
36 &  23.2 &   9.50 \\
37 &  23.5 &   8.41 \\
38 &  23.9 &   9.06 \\
39 &  24.5 &  11.47 \\
40 &  25.1 &  19.38 \\
41 &  26.1 &  32.45 \\
42 &  26.4 &  34.67 \\
43 &  26.8 &  34.68 \\
44 &  27.2 &  31.35 \\
45 &  28.2 &  19.44 \\
46 &  28.6 &  17.12 \\
47 &  29.2 &  16.76 \\
48 &  29.5 &  18.05 \\
49 &  30.5 &  28.12 \\
50 &  30.9 &  33.55 \\
\bottomrule
\end{tabular}
\\Neon
\begin{tabular}{lrr}
\toprule
{} &  U[V] &  Ua[V] \\
\midrule
0  &   0.0 &   0.86 \\
1  &   1.1 &   0.85 \\
2  &   2.3 &   0.97 \\
3  &   3.5 &   0.85 \\
4  &   5.1 &   1.06 \\
5  &   6.0 &   1.45 \\
6  &   7.4 &   1.86 \\
7  &   8.8 &   2.13 \\
8  &   9.7 &   2.33 \\
9  &  10.6 &   2.55 \\
10 &  11.3 &   2.63 \\
11 &  12.4 &   2.84 \\
12 &  13.7 &   3.02 \\
13 &  14.9 &   3.23 \\
14 &  15.8 &   3.41 \\
15 &  16.8 &   3.51 \\
16 &  18.1 &   3.12 \\
17 &  20.3 &   1.67 \\
18 &  20.9 &   1.57 \\
19 &  21.5 &   1.42 \\
20 &  22.0 &   1.23 \\
21 &  22.7 &   1.72 \\
22 &  23.7 &   3.42 \\
23 &  25.0 &   6.11 \\
24 &  25.3 &   6.52 \\
25 &  26.0 &   7.76 \\
26 &  28.0 &  10.03 \\
27 &  28.5 &  10.44 \\
28 &  29.1 &  10.80 \\
29 &  30.0 &  11.07 \\
30 &  31.1 &  11.78 \\
31 &  32.6 &  12.37 \\
32 &  33.7 &  12.69 \\
33 &  34.5 &  11.42 \\
34 &  36.6 &   5.56 \\
35 &  37.5 &   3.34 \\
36 &  38.5 &   2.02 \\
37 &  39.5 &   2.04 \\
38 &  40.6 &   4.90 \\
39 &  42.3 &  10.09 \\
40 &  43.3 &  12.71 \\
41 &  44.3 &  14.85 \\
42 &  45.3 &  16.33 \\
43 &  46.5 &  17.88 \\
44 &  47.0 &  18.27 \\
45 &  47.5 &  18.77 \\
46 &  48.0 &  19.27 \\
47 &  49.0 &  19.96 \\
48 &  49.5 &  20.21 \\
49 &  50.0 &  20.52 \\
50 &  51.0 &  20.49 \\
51 &  51.5 &  19.58 \\
52 &  54.0 &  11.11 \\
53 &  56.0 &   5.55 \\
54 &  56.5 &   5.53 \\
55 &  57.0 &   6.43 \\
56 &  57.5 &   7.48 \\
57 &  60.0 &  14.01 \\
58 &  63.0 &  22.51 \\
59 &  64.1 &  24.77 \\
60 &  65.0 &  26.57 \\
61 &  66.0 &  28.18 \\
62 &  66.6 &  29.21 \\
63 &  67.1 &  29.76 \\
64 &  67.6 &  30.31 \\
65 &  68.1 &  30.14 \\
66 &  68.6 &  29.61 \\
67 &  70.0 &  26.33 \\
\bottomrule
\end{tabular}
\section{Analiza niepewności}

\section{Wnioski}
\paragraph{}...


\end{document}
