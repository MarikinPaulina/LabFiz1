\documentclass[a4paper,10pt]{article}
\usepackage[margin=1in]{geometry}
\usepackage{polski}
\usepackage[utf8x]{inputenc}
\usepackage[unicode]{hyperref}
\usepackage{amssymb}
\usepackage{xifthen}
\usepackage[fleqn]{amsmath}
\usepackage{todonotes}
\usepackage{graphicx}
\usepackage{float}
\usepackage{fullpage}
\usepackage{epstopdf}
\usepackage{multirow}
\usepackage{subfig}
\usepackage[europeanresistors,americaninductors]{circuitikz}
\usetikzlibrary{patterns}
\newcommand{\withtodo}{0}


\def\arraystretch{1.2}


\begin{document}

\begin{table}
  \centering
  \def\arraystretch{1.5}
    \begin{tabular}{|l|l|l|l|} \hline
    Wydział:           & \multicolumn{2}{l|}{Dzień:Poniedziałek 14-17}    &Zespół:  \\
    Fizyki             &    \multicolumn{2}{l|}{Data: 20.03.2017}         &8             \\\hline
    Imiona i nazwiska: &Ocena z przygotowania:  &Ocena ze sprawozdania:   &Ocena końcowa: \\
    Marta Pogorzelska  &                        &                         &                \\
    Paulina Marikin    &                        &                         &\\\hline
    \multicolumn{2}{|l|}{Prowadzący:                 } &\multicolumn{2}{l|}{Podpis:             }  \\\hline
  \end{tabular}
\end{table}

\title{Ćwiczenie 43:\\Wyznaczanie $\frac{c_p}{c_v}$ dla powietrza metodą rezonansu akustycznego}
\date{}
\maketitle

\section{Cel badań}
Doświadczenie miało na celu wyznaczenie współczynnika adiabaty dla powietrza.

\section{Wstęp teoretyczny}
$\kappa$ jest współczynnikiem w równaniu adiabaty, zależnym od budowy molekuł danego gazu i powiązanym z nim równaniem:

\begin{equation}
  \kappa = \frac{c_p}{c_V}
\end{equation}
$c_p$ - ciepło właściwe przy stałym ciśnieniu, $c_V$ - ciepło właściwe przy stałej objętości.
\\\\W tym doświadczeniu jego wartość dla powietrza została wyznaczona metodą Laplace'a, wiążącą równania termodynamiczne z zachowaniem fali akustycznej,
której ruch jest szeregiem przemian adiabatycznych. Można więc do jego opisania stosować równanie adiabaty, z którego, w połączeniu z równaniem
falowym otrzymujemy:

\begin{equation}
  \kappa = \frac{v^2 \rho}{p}
\end{equation}
Prędkość fali została zmierzona pośrednio na podstawie równości $v = \lambda f$ co wstawione do poprzedniego równania daje nam finalny wzór:
\begin{equation}
  \kappa = \frac{\lambda^2 f^2 M}{kT}
\end{equation}

\section{Opis układu i metody pomiarowej}
Użyte przyrządy:
\begin{itemize}
  \item oscyloskop
  \item miarka z podziałką 1mm
  \item głośnik
  \item mikrofon na ruchomym tłoku
  \item rurka z plexi
  \item generator ze wzmacniaczem
  \item termometr z podziałką $2^\circ C$
\end{itemize}
W celu wyznaczenia kolejnych długości fali mierzone były odległości między kolejnymi węzłami fali stojącej, utworzonej poprzez poruszanie tłokiem z
doczepionym mikrofonem i obserwację obrazu na oscyloskopie na ustawieniu X-Y. %TODO: lambda = 2a
Zamiast okresu dla każdej z fal została zmierzona częstotliwość $\omega = 2 \pi T$, mierzona jako odległość między kolejnymi maksimami fali stojącej na
obrazie z oscyloskopu. Temperatura została zmierzona raz, po wykonaniu pozostałych pomiarów.

\section{Analiza pomiarów}

\section{Analiza niepewności}

\section{Wnioski}
\paragraph{}...


\end{document}
