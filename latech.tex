\documentclass[a4paper,10pt]{article}
\usepackage[margin=1in]{geometry}
\usepackage{polski}
\usepackage[utf8x]{inputenc}
\usepackage[unicode]{hyperref}
\usepackage{amssymb}
\usepackage{xifthen}
\usepackage[fleqn]{amsmath}
\usepackage{todonotes}
\usepackage{graphicx}
\usepackage{float}
\usepackage{fullpage}
\usepackage{epstopdf}
\usepackage{multirow}
\usepackage{subfig}
\usepackage[europeanresistors,americaninductors]{circuitikz}
\usetikzlibrary{patterns}
\newcommand{\withtodo}{0}


\def\arraystretch{1.2}


\begin{document}

\begin{table}
  \centering
  \def\arraystretch{1.5}
    \begin{tabular}{|l|c|r|} \hline
    Laboratorium Fizyki I                  & \multicolumn{2}{l|}{Data wykonania ćwiczenia:    16.12.15  }  \\
    Wydział: Fizyki                        & \multicolumn{2}{l|}{Godziny: 8 - 13}                      \\\hline
    Imiona i nazwiska wykonawców: & Data złożenia sprawozdania: & Nr ćwiczenia:   \\
    imię nazwisko, imię nazwisko                        &  21.12.15                &                          \\\hline
    Prowadzący: imię nazwisko              &Ocena końcowa:                      &                                       \\\hline
  \end{tabular}
\end{table}

\begin{figure}
    \centering
    \includegraphics[width=0.1\textwidth]{howdy.jpg}
    \caption{Howdy, pardner! I'm your best friend!}
    \label{flowyyyyyyyy}
\end{figure}

\section{Cel ćwiczenia}
\paragraph{lorem ipsusdsdsdsdB}

w tekście $x^2 + y^2$

jako oddzielna linijka
$$ x^2 + y^2 = z^2 $$

\begin{equation}
 \frac{1}{N}\frac{dn_E}{dE}=\Big(\frac{4}{\pi}\Big)^{1/2}(kT)^{-3/2}E^{1/2}\exp{\Big(\frac{-E}{kT}\Big)}
\end{equation}

\begin{equation}
    \ln{(\frac{I_a}{I_{a0}})} = \frac{-e}{kT}U_a
\end{equation}


\section{Wstęp teoretyczny}
mech

\{ % to jest
\section{Opis układu i metody pomiarowej}

\paragraph{}...

\section{Wyniki pomiarów}...

\begin{tabular}{|l|r|r|r|r|}
\toprule
{} & \multicolumn{2}{l}{Seria 1} & \multicolumn{2}{l}{Seria 2} \\
{} &  $U_{a}[\text{mV}]$ &  $U_{a}[\text{m}V]$ &  U$_{a2}$[mV] &  Ia2[$\mu$A] \\
\midrule
0  &     -0.0 &        15 &     -0.7 &       150 \\
1  &    -10.2 &        14 &    -14.3 &       140 \\
2  &    -16.9 &        13 &    -30.2 &       130 \\
3  &    -24.9 &        12 &    -47.2 &       120 \\
4  &    -32.9 &        11 &    -63.5 &       110 \\
5  &    -42.4 &        10 &    -79.5 &       100 \\
6  &    -51.9 &         9 &    -98.3 &        90 \\
7  &    -63.7 &         8 &   -116.5 &        80 \\
8  &    -75.0 &         7 &   -136.4 &        70 \\
9  &    -89.3 &         6 &   -157.8 &        60 \\
10 &   -104.6 &         5 &   -181.2 &        50 \\
\bottomrule
\end{tabular}
\section{Wnioski}
\paragraph{}...


\end{document}
