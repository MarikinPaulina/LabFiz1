\documentclass[a4paper,10pt]{article}
\usepackage[margin=1in]{geometry}
\usepackage{polski}
\usepackage[utf8x]{inputenc}
\usepackage[unicode]{hyperref}
\usepackage{amssymb}
\usepackage{xifthen}
\usepackage[fleqn]{amsmath}
\usepackage{todonotes}
\usepackage{graphicx}
\usepackage{float}
\usepackage{fullpage}
\usepackage{epstopdf}
\usepackage{multirow}
\usepackage{subfig}
\usepackage[europeanresistors,americaninductors]{circuitikz}
\usetikzlibrary{patterns}
\newcommand{\withtodo}{0}

\def\arraystretch{1.2}

\begin{document}

\begin{table}
  \centering
  \def\arraystretch{1.5}
    \begin{tabular}{|l|l|l|l|} \hline
    Wydział:           & \multicolumn{2}{l|}{Dzień:Poniedziałek 14-17}    &Zespół:  \\
    Fizyki             &    \multicolumn{2}{l|}{Data: 20.03.2017}         &8             \\\hline
    Imiona i nazwiska: &Ocena z przygotowania:  &Ocena ze sprawozdania:   &Ocena końcowa: \\
    Marta Pogorzelska  &                        &                         &                \\
    Paulina Marikin    &                        &                         &\\\hline
    \multicolumn{2}{|l|}{Prowadzący:                 } &\multicolumn{2}{l|}{Podpis:             }  \\\hline
  \end{tabular}
\end{table}

\section{Cel badań}

\section{Wstęp teoretyczny}

\section{Metoda przeprowadzenia badań i pomiarów, materiały, aparatura}

\section{Wyniki pomiarów}

\section{Opracowanie pomiarów}

\begin{tabular}{|l|r|r|r|r|r|r|}
\hline
\multicolumn{7}{|c|}{Glin} \\\hline
{} &     $x$ &      S$s$ &   $\mu$x &     $x$ &N($\mu_N$) &$\lnN(\mu_{lnN})$ \\\hline
20 &  20.041 &  0.002214 &  0.006183 &  20.041 &  595.0 &  6.388561 \\\hline
15 &  14.809 &  0.005187 &  0.007761 &  14.809 &  715.0 &  6.572283 \\\hline
10 &  10.008 &  0.002366 &  0.006240 &  10.008 &  731.0 &  6.594413 \\\hline
5  &   5.020 &  0.003464 &  0.006733 &   5.020 &  754.0 &  6.625392 \\\hline
\multicolumn{7}{|c|}{Miedź}                                         \\\hline
{} &     $x$ &      S$s$ &   $\mu$x &     $x$ &N($\mu_N$) &$\lnN(\mu_{lnN})$ \\\hline
20 &  20.112 &  0.002366 &  0.006875 &  20.112 &  281.0 &  5.638355 \\\hline
17 &  16.966 &  0.002530 &  0.006933 &  16.966 &  329.0 &  5.796058 \\\hline
15 &  15.080 &  0.002000 &  0.006758 &  15.080 &  393.0 &  5.973810 \\\hline
12 &  12.103 &  0.002470 &  0.006911 &  12.103 &  465.0 &  6.142037 \\\hline
10 &  10.972 &  0.003688 &  0.007434 &  10.972 &  484.0 &  6.182085 \\\hline
7 &   7.045 &  0.002915 &  0.007083 &   7.045 &  555.0 &  6.318968 \\\hline
5 &   4.849 &  0.003592 &  0.007387 &   4.849 &  626.0 &  6.439350 \\\hline
2 &   1.948 &  0.002757 &  0.007019 &   1.948 &  754.0 &  6.625392 \\\hline
\multicolumn{7}{|c|}{Ołów}                                          \\\hline
{} &     $x$ &      S$s$ &   $\mu$x &     $x$ &N($\mu_N$) &$\lnN(\mu_{lnN})$ \\\hline
20 &  19.953 &  0.004012 &  0.007600 &  19.953 &   89.0 &  4.488636 \\\hline
17 &  16.910 &  0.002828 &  0.007047 &  16.910 &  122.0 &  4.804021 \\\hline
15 &  14.840 &  0.004000 &  0.007594 &  14.840 &  157.0 &  5.056246 \\\hline
12 &  11.895 &  0.002915 &  0.007083 &  11.895 &  267.0 &  5.587249 \\\hline
10 &   9.976 &  0.002530 &  0.006933 &   9.976 &  315.0 &  5.752573 \\\hline
7 &   6.951 &  0.002214 &  0.006824 &   6.951 &  363.0 &  5.894403 \\\hline
5 &   5.037 &  0.003479 &  0.007333 &   5.037 &  504.0 &  6.222576 \\\hline
2 &   1.845 &  0.002550 &  0.006940 &   1.845 &  711.0 &  6.566672 \\\hline
\end{tabular}


\section{Analiza niepewności}

 \section{Wnioski}

\end{document}
