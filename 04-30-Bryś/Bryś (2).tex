\documentclass[a4paper,10pt]{article}
\usepackage[margin=1in]{geometry}
\usepackage{polski}
\usepackage[utf8x]{inputenc}
\usepackage[unicode]{hyperref}
\usepackage{amssymb}
\usepackage{xifthen}
\usepackage[fleqn]{amsmath}
\usepackage{todonotes}
\usepackage{graphicx}
\usepackage{float}
\usepackage{fullpage}
\usepackage{epstopdf}
\usepackage{multirow}
\usepackage{subfig}
\usepackage[europeanresistors,americaninductors]{circuitikz}
\usetikzlibrary{patterns}
\newcommand{\withtodo}{0}


\def\arraystretch{1.2}


\begin{document}

\begin{table}
  \centering
  \def\arraystretch{1.5}
    \begin{tabular}{|l|l|l|l|} \hline
    Wydział:           & \multicolumn{2}{l|}{Dzień:Poniedziałek 14-17}    &Zespół:  \\
    Fizyki             &    \multicolumn{2}{l|}{Data: 20.03.2017}         &8             \\\hline
    Imiona i nazwiska: &Ocena z przygotowania:  &Ocena ze sprawozdania:   &Ocena końcowa: \\
    Marta Pogorzelska  &                        &                         &                \\
    Paulina Marikin    &                        &                         &\\\hline
    \multicolumn{2}{|l|}{Prowadzący:                 } &\multicolumn{2}{l|}{Podpis:             }  \\\hline
  \end{tabular}
\end{table}

\title{Ćwiczenie 30:\\Odbicie światła od powierzchni dielektryka}
\date{}
\maketitle

\section{Cel badań}
Celem doświadczenia było zweryfikowanie poprawności prawa Snella i prawa Malusa oraz wyznaczenie
kąta granicznego, kąta Brewstera i wspówłczynnika załamania badanego dielektryka.

\section{Wstęp teoretyczny}
\subsection{Prawo Snella}
Fala elektromagnetyczna na granicy ośrodków ulega dwóm zjawiskom: załamaniu i odbiciu, gdzie fala załamana jest częścią fali, która zmieniła ośrodek, zaś fala odbita częścią pozostałą w pierwotnym ośrodku. Kąty pod jakimi rozchodza się te fale (mierzone do normalnej - osi prostopadłej do płaszczyzny odbicia) są ze sobą powiązane przez prawo Snella:

\begin{equation}
n_1\sin{\alpha} = n_2\sin{\beta}
\end{equation}

,gdzie kąt $\alpha$ jest kątem odbicia równym kątowi padania, kąt $\beta$ to kąt załamania, zaś $n_1$ i $n_2$ to współczynniki załamania definiowane $n = \frac{c}{v}$ ,gdzie v - prędkość fali elektromagnetycznej w danym ośrodku. Po przekształceniu wzoru (1):

\begin{equation}
n_2 = n_1 \frac{\sin{\alpha}}{\sin{\beta}}
\end{equation}

można na podstawie prawa Snella wyznaczyć eksperymentalnie współczynnik załamania danego ośrodka.

\begin{figure}[H]
\centering
\includegraphics[width=0.5\textwidth]{./snellius.png}
  \caption{prawo Snella.}
  \label{}
\end{figure}

\subsection{Kąt Brewstera}
Na potrzeby badanego zjawiska falę spolaryzowaną rozważymy jako nałożenie się dwóch fal o prostopadłych polaryzacjach, z wektorem pola elektrycznego prostopadłym($\sigma$) i równoległym($\pi$) do płaszczyzny padania. Fala odbita złożona jest z ułamków obu tych fal zależnych od kąta padania, przy czym składowa polaryzacji $\sigma$ rośnie monotonicznie wraz ze wzrostem kąta padania, zaś należąca do polaryzacji $\pi$
początkowo maleje i wzrasta dopiero po osiągnięciu 0. Kąt w którym polaryzacja $\pi$ fali odbitej nie występuje nosi nazwę kąta Brewstera.
Współczynnik tej polaryzacji jest opisany przes wzór Fresnela:
\begin{equation}
  R = \frac{\tg^2({\alpha - \beta})}{\tg^2({\alpha + \beta})}
\end{equation}
na podstawie którego można powiedzieć, że kąt Brewstera przypada gdy $\alpha + \beta=\frac{\pi}{2}$, więc $\beta = \frac{\pi}{2}-\alpha$, co
po podstawieniu do prawa Snella daje:
\begin{equation}
  n_1 \sin{\alpha_B} = n_2 \cos{\alpha_B}
\end{equation}
i dalej:
\begin{equation}
  n_2 = n_1 \tg{\alpha_B}
\end{equation}

\begin{figure}[H]
\centering
\includegraphics[width=0.5\textwidth]{./brewster.png}
  \caption{kąt Brewstera.}
  \label{}
\end{figure}

\subsection{Kąt graniczny}
Jeżeli w pierwotnym ośrodku światło poruszało się szybciej, to dla dużych kątów padania kąt załamania przekroczył by $\frac{\pi}{2}$. W takiej sytuacji zjawisko załamania nie występuje i cała fala jest odbita. Kąt padania, dla którego kąt załamania wynosi dokładnie $\frac{\pi}{2}$, jest nazywany kątem granicznym. Co więcej, ponieważ $\sin{\frac{\pi}{2}} = 1$, to dla kąta granicznego zachodzi rówonść:

\begin{equation}
 n_1 = \frac{n_2}{\sin{\alpha_{gr}}}
\end{equation}

\begin{figure}[H]
\centering
\includegraphics[width=0.5\textwidth]{./graniczny.png}
  \caption{kąt graniczny.}
  \label{}
\end{figure}

\subsection{Prawo Malusa}
Jeśli kierunek natężnia pola elektrycznego w fali jest stały lub zmienia się w sposób ściśle określony, to jest ona spolaryzowana liniowo. Po ponownym spolaryzowaniu takiej fali przepuszczona pozostanie tylko ta jej część, dla której pole elektryczne było współosiowe z osią polaryzatora. Dla $\theta$ - kąt między osią polaryzacji i osią polaryzatora - zachodzi zależność:

\begin{equation}
  E = E_0 \cos{\theta}
\end{equation}

lub, przekształcając na natężenie wiązki światła:

\begin{equation}
  I=I_0 \cos^2{\theta}
\end{equation}


\section{Opis układu i metody pomiarowej}
Użyte przyrządy
\begin{itemize}
  \item laser
  \item stolik goniometryczny
  \item fotodetektor z miernikiem
  \item dwa polaryzatory
  \item dielektryk w kształcie półwalca
  \item tarcza osłonowa
\end{itemize}

\subsection{Prawo Snella i kąt Brewstera}
Najpierw wykonane zostały pomiary kątów załamania przy przejściu z powietrza do badanej płytki dla kąta padania równego $10^\circ$ i kolejnych
aż do $80^\circ$ za każdym razem zwiększając kąt o $10^\circ$. Kąt padania był zmieniany poprzez poruszanie stolika goniometrycznego. Do
pomiaru kąta załamania użyto fotodetektora, z którego położenia (przy padającej nań wiązce) szczytywany był kąt załamania. Następnie między
dwoma pomiarami wyznaczony został przedział występowania kąta Brewstera i zmieniając kąt padania co $1^\circ$ szukany był kąt, dla którego
suma kątów padania i załamania była równa $90^\circ$.

\subsection{Kąt graniczny}
W celu wyznaczenia kąta granicznego użyto dodatkowo tarczy osłonowej i ustawiono dielektryk tak, by początkowo wiązka przechodziła przez
niego i  padała na tarczę, a początkowy kąt padania był równy 0.
Następnie zwiększając kąt szukano takiego, dla którego wiązka załamana propaguje się na powierzchni granicy ośrodków i
następuje zjawisko całkowitego wewnętrznego odbicia wiązki.

\subsection{Prawo Malusa}
W ostatnim etapie doświadczenia ustawiono fotodetektor na linii wiązki, a między nim i źródłem światła umiejscowiono dwa polaryzatory.
Poprzez obracanie polaryzatorem względem osi obrotu, położonym dalej od źródła światła  (dalej nazywanym analizatorem) znaleziono kąt,
dla którego wskazania miernika fotoprądu były najwyższe. Dany kąt analizatora został uznany za zerowy. Kolejne pomiary natężenia wiązki
były wykonywane co $15^\circ$ obrotu analizatora aż do osiągnięcia $90^\circ$.

\section{Wyniki i analiza pomiarów}
\subsection{Prawo Snella i kąt Brewstera}

\begin{tabular}{lrrrrrrrr}
\toprule
\hline
{}&$\alpha[^\circ]$&$\beta[^\circ]$&u($\alpha[^\circ]$)&u($\beta[^\circ]$)&$\sin{\alpha}$&$\sin{\beta}$&u($\sin{\alpha}$)&u($\sin{\beta}$) \\
\midrule
\hline
0 & 10.0 &  6.0 & 1.118 & 1.118 & 0.174 & 0.105 & 0.019 & 0.019 \\
1 & 20.0 & 13.0 & 1.118 & 1.118 & 0.342 & 0.225 & 0.018 & 0.019 \\
2 & 30.0 & 19.5 & 1.118 & 1.118 & 0.500 & 0.334 & 0.017 & 0.018 \\
3 & 40.0 & 25.0 & 1.118 & 1.118 & 0.643 & 0.423 & 0.015 & 0.018 \\
4 & 50.0 & 31.0 & 1.118 & 1.118 & 0.766 & 0.515 & 0.013 & 0.017 \\
5 & 60.0 & 35.0 & 1.118 & 1.118 & 0.866 & 0.574 & 0.010 & 0.016 \\
6 & 70.0 & 38.5 & 1.118 & 1.118 & 0.9397 & 0.6225 & 0.0067 & 0.0153 \\
7 & 80.0 & 40.0 & 1.118 & 1.118 & 0.9848 & 0.6428 & 0.0034 & 0.0149 \\
\hline
\bottomrule
\end{tabular}

\begin{figure}[H]
  \includegraphics{./snella.png}
  \caption{Wykres nr 1: Zależność sinusa kąta padania od sinusa kąta załamania wiązki światła przechodzącej przez dielektryk.}
\end{figure}
W celu sprawdzenia prawa Snella wykreślono zależności z równania (2), tak by przedstawiała sobą zależność linową. Prosta została dopasowana
przy użyciu funkcji \emph{polyfit} biblioteki \emph{numpy} w Pythonie. Za niepewność wyznaczonego w ten sposób współczynnika wzięto pierwiastek zwróconej
przez funkcję kowariancji.
\\Wyznaczony w pomiarach kąt Brewstera wynosi: $\alpha_B = 56.25^\circ(1.1)$
\subsection{Kąt graniczny}
Wyznaczony kąt graniczny wynosi: $\alpha_g = 44.0(1.1)^\circ$

\subsection{Prawo Malusa}
\begin{tabular}{lrrrrr}
\toprule
\hline
{} &$\theta[^\circ]$&u($\theta[^\circ]$)&I&$|z|$&u(I) \\
\midrule
\hline
0 &  0.0 & 1.118 & 0.900[mA] & 1[mA] & 0.025[mA] \\
1 & 15.0 & 1.118 & 0.820[mA] & 1[mA] & 0.025[mA] \\
2 & 30.0 & 1.118 & 0.620[mA] & 1[mA] & 0.025[mA] \\
3 & 45.0 & 1.118 & 0.400[mA] & 1[mA] & 0.025[mA] \\
4 & 60.0 & 1.118 & 0.1800[mA] & 0.3[mA] & 0.0075[mA] \\
5 & 75.0 & 1.118 & 43.0[$\mu$A] & 100[$\mu$A] & 2.5[$\mu$A] \\
6 & 90.0 & 1.118 & 1.4[$\mu$A] &  3[$\mu$A] &  0.075[$\mu$A] \\
\bottomrule
\hline
\end{tabular}

\begin{figure}[H]
  \includegraphics{./malus.png}
  \caption{}
\end{figure}
W celu weryfikacji prawa Malusa wykreślono wykres zależności z równania (8), tak by przedstawiał zależność liniową.
Prosta została dopasowana przy użyciu funkcji \emph{polyfit} biblioteki \emph{numpy} w Pythonie.
\subsection{Współczynnik załamania}
\begin{tabular}{|l|c|c|}
  \hline
  Metoda & Współczynnik &Niepewność \\
  pomiarowa & załamania & rozszerzona\\\hline
  Prawo Snella & 1.498 & 0.044 \\\hline
  Kąt Brewstera& 1.497 & 0.022 \\\hline
  Kąt graniczny& 1.440 & 0.058 \\\hline
\end{tabular}

\section{Analiza niepewności}

\section{Wnioski}
\paragraph{}...


\end{document}
